\chapter{Documentación}
\lettrine[lines=1,slope=4pt,findent=0pt]{U}{}na vez introducida la arquitectura de pizarra y la estructura básica de la plataforma, daremos paso a una documentación detallada de la misma. Para abordar la documentación de la plataforma ha sido necesario previamente la obtención de los requisitos, para después, una vez desarrollados y clasificados convenientemente, dar lugar a la realización de los diagramas UML correspondientes.

\section{Obtención de los requisitos}
Para abordar la obtención de los requisitos hemos partido de una posible aplicación que pudiera ser implementada con nuestra librería para, a partir de ahí, generalizar los mismos y extrapolarlos a la librería.\\

La aplicación tomada como referencia para la extracción de requisitos permite a un profesor, así como a sus alumnos subir y modificar archivos en una pizarra externa, permitiendo así la cómoda realización de trabajos en grupo por parte de los alumnos y un seguimiento por parte del profesor.\\

Pensar de un modo más aplicado nos ha permitido extraer requisitos con más facilidad, pensando en qué cosas serían necesarias para el buen funcionamiento de la aplicación.\\

Una vez extraídos los requisitos se han clasificado según lo siguiente:

\begin{itemize}
	\item \textbf{Funcionales: }Son declaraciones de los servicios que debe proporcionar el sistema. Especifica la manera en que éste debe reaccionar a determinadas entradas. Especifica cómo debe comportarse el sistema en situaciones particulares.
	\item \textbf{No funcionales: } Restricciones de los servicios o funciones ofrecidas por el sistema (fiabilidad, tiempo de respuestas, capacidad de almacenamiento, etc.). Generalmente se aplican al sistema en su totalidad. Surgen de las necesidades del usuario debido a restricciones de presupuesto, políticas de la organización, necesidad de interoperatividad, etc.
	\begin{itemize}
		\item Del producto:
		\item Organizacionales:
		\item Requisitos externos:
	\end{itemize}
\end{itemize}

\subsection{Requisitos}