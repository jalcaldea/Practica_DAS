\chapter{Introducción}
\lettrine[lines=1,slope=4pt,findent=0pt]{E}{}n la segunda parte de la memoria describiremos a nivel de arquitectura y diseño una plataforma que implemente la arquitectura de pizarra. \\

En primer lugar debemos describir qué es exactamente una arquitectura de pizarra, para seguidamente poder analizar en detalle la plataforma a crear.\\

Una arquitectura de pizarra es un modelo arquitectónico para el software utilizada en los siguientes casos:

\begin{itemize}
	\item Problemas en los que no existe una solución analítica.
	\item Problemas en los que, aunque exista solución analítica, es inviable por la dimensión del espacio de búsqueda.
	\item Problemas extremadamente complejos en términos cognitivos.
\end{itemize}

La arquitectura de pizarra consta principalmente de dos componentes:

\begin{itemize}
	\item \textbf{Pizarra: }Instrumento de control o estructura de datos que representa el estado actual de la plataforma.
	\item \textbf{Agente: }Elementos funcionales independientes entre sí que operan sobre la pizarra.
\end{itemize}

La función de los agentes es, basándose en el contenido de la pizarra, realizar la tarea que se les asigna y escribir sobre ella sus resultados.\\

La función principal de la pizarra es coordinar la interacción de los agentes con la pizarra y la comunicación entre ellos.
