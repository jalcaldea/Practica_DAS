\chapter{Manual de uso de la librería}
Una vez especificados los requisitos de nuestra pizarra, además de los UML necesarios que describen su funcionamiento, incorporamos un breve manual de usuario con el que un programador ajeno podría construir fácilmente un agente o grupo de agentes que utilicen la pizarra para comunicarse.

\section{Consideraciones previas}
\color{red}QUE ESTÉ TODO ACTUALIZADO, BIEN CONFIGURADO Y Y BLA BLA BLA
\color{black}
\section{Identificación e ingreso}
En primer lugar un usuario tendrá que conectarse mediante su "USERID" y "PASS". Si ambas son correctas, ya se habrá conectado a la pizarra en su nivel correspondiente.
Una vez dentro, podrá consultar las estadísticas (tanto a nivel de usuario como generales), buscar y comparar archivos y BLA BLA BLA.
A continuación se especifican de manera más específica el uso de las instrucciones in y out
\section{Cargar}
\section{Guardar}


