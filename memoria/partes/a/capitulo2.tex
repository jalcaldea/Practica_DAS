\chapter{Descripción de la arquitectura de virtualización}

\lettrine[lines=1,slope=4pt,findent=0pt]{U}{}na vez introducido el vocabulario básico y tras haber indagado un poco más en la materia vamos a centrarnos en la arquitectura como tal.\\

\section{Tipos de Virtualización}
En el apartado anterior ya se pudo vislumbrar que existen diferentes formas de virtualización, y por ello, nosotros las vamos a nombrar todos los tipos y haremos, para cada uno, una pequeña descripción:

\subsection{Arquitectura del 


\section{Componentes}

El esquema básico es el siguiente:

\begin{center}
\begin{tikzpicture}
\path (0,0) coordinate (A) {}
        (1,1) coordinate (B) {}
  (3,0) coordinate (C) {};
\draw[dashed] (A) -- (B) -- (C);
\end{tikzpicture}
\end{center}

hipervisor}
\subsection{La arquitectura de Xen}
\subsection{Traducción binaria en la virtualización completa}
\subsection{Virtualización completa}
\subsection{Virtualización con un host}
\subsection{Paravirtualización}

\section{Componentes}

Como bien se ha introducido antes, en la arquitectura de hipervisor

\subsection{Máquina Virtual}
El esquema básico es el siguiente:

\begin{center}
\fbox{\parbox{70mm}{AQUI VA LA FIGURA QUE SE PUEDE ENCONTRAR EN:\\ \\
\url{http://upload.wikimedia.org/wikipedia/commons/3/3d/VM\_Monitor\_IBM\_VM\_370.jpg}\\ \\
\textcolor{red}{HAY QUE HACERLA}}}
\end{center}