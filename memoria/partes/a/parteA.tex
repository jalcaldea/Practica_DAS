\chapter{Introducción}
\lettrine[lines=1,slope=4pt,findent=0pt]{E}{}n esta primera parte de la memoria nos vamos a encargar de hacer una descripción a nivel arquitectónico de los sistemas de virtualización de las aulas de la Universidad Rey Juan Carlos.\\

\noindent Como nos vamos a centrar en los ordenadores de las aulas, éstos se encuentran virtualizados con VMware~\cite{vmware}\\

\noindent En principio nos centraremos en las arquitecturas de virtualización de una forma genérica y seguidamente analizaremos específicamente el sistema utilizado en la URJC.


\chapter{Definición arquitectura virtualización}

\lettrine[lines=1,slope=4pt,findent=0pt]{L}{}a de máquina virtual o intérprete se puede descomponer, a grandes rasgos, en cuatro componentes:
\section{motor de simulación o interpretación}
\section{memoria que contiene el código a interpretar}
\section{representación del estado de la interpretación}
\section{representación del estado del programa que se esta
simulando}
