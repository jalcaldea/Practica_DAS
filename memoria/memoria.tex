\documentclass[12pt,a4paper,titlepage,twoside]{book}
\usepackage[utf8]{inputenc}
\usepackage[spanish, es-lcroman]{babel}
\usepackage{amsmath}
\usepackage{amsthm}
\usepackage{amsfonts}
\usepackage{amssymb}
\usepackage{graphicx}
\usepackage{lettrine}
\usepackage{float}
\usepackage{color}
\usepackage[top=2cm,bottom=4cm,outer=1.5cm,inner=3cm]{geometry}
\usepackage[nottoc]{tocbibind}	%Para mostrar la bibliografia en el índice
\usepackage{tikz}
\usepackage[refpages]{gloss} % Para el glosario
\usepackage{emptypage} % Para eliminar encabezado de las páginas en blanco
\usepackage{chngcntr}
\usepackage{etoolbox}
\usepackage{./tikz-uml} % De esta forma no hay que añadir paquete al directorio de LaTeX


% Aquí van algunos comandos para no hacer demasiado largo el preámbulo.


% Separa la lista de figuras por partes
% % % % % % % % % % % % % % % % % % % % % % % % % % % % % % % % % % % % % % % %
%
% NOTA IMPORTANTE: Este comando SIEMPRE debe ir antes de la inclusión del paquete hyperref.
%
% % % % % % % % % % % % % % % % % % % % % % % % % % % % % % % % % % % % % % % % 
% Adaptado de:
%http://tex.stackexchange.com/questions/52746/include-chapters-in-list-of-figures-with-titletoc
\makeatletter
\def\thisparttitle{}\def\thispartnumber{}
\newtoggle{noFigs}

\apptocmd{\@part}%
  {\gdef\thisparttitle{#1}\gdef\thispartnumber{\thepart}%
    \global\toggletrue{noFigs}}{}{}

% the figure environment does the job: the first time it is used after a \chapter command, 
% it writes the information of the chapter to the LoF
\AtBeginDocument{%
  \AtBeginEnvironment{figure}{%
    \iftoggle{noFigs}{
      \addtocontents{lof}{\protect\contentsline {chapter}%
        {\protect\numberline {\thispartnumber} {\thisparttitle}}{}{} }
      \global\togglefalse{noFigs}
    }{}
  }%
}

\makeatother

% Reinicia el contador a 1 en cada parte
\makeatletter
\@addtoreset{chapter}{part} 
\makeatother
% Aquí van todos los diagramas UML
\usepackage{tikz-uml}

% Diagrama de clases
\newcommand{\clases}{
\begin{figure}
\centering
\begin{tikzpicture}

% Clase Pizarra
\umlclass{Pizarra}{
  - estado : Estado \\ 
  - usuarios : ArrayList$<$Usuario$>$\\
  - estadisticas : ArrayList$<$Estadisticas$>$\\
  - niveles : ArrayList$<$Nivel$>$\\
  - conf : Configuracion\\
}{ 
  + mostrarEstadisticas(String userid, Usuario user):String\\
  + mostrarEstadisticasGenerales(Usuario user):String\\
  + escribir(String direccion, Archivo archivo, Usuario user)\\
  + leer(String direccion, Usuario user):Archivo\\
  + escrituraLectura(String direccion, Archivo archivo, Usuario user): Archivo\\
  + mostrarEstado(Usuario user):Usuario\\
  + buscar(String nombre, Usuario user):Estado\\
  + comparar(Archivo arch1, Archivo arch2, Usuario user):Archivo\\
  + crearCarpeta(String direccion, String nombre, Usuario user): Bool\\
  + crearUsuario(String nombre, String userid, String pass, Nivel nivel, Usuario user)\\
  + gestionarPermisos(Nivel nivel, Bool[] permisos, Usuario user): Bool\\
  + configurarPizarra(Configuracion config, Usuario user): Bool\\
  + getUsuario(String userid):Usuario\\
  - setEstado(Estado)\\
  - setUsuarios(ArrayList$<$Usuario$>$)\\
  - setEstadisticas(ArrayList$<$Estadisticas$>$)\\
  - comprobarPermisos(Usuario user, int permiso):Bool
}

% Clase Agente
\umlclass[x=0, y=-14]{Agente}{
  - userid : String\\
  - pass : String\\
  - pizarra : Pizarra\\
  - usuario : Usuario\\
}{
  + Agente(String userid)\\
  + Agente(Pizarra pizarra)\\
  + iniciarSesion(String userid, String pass):Bool\\
  + mostrarEstadisticas(String userid)\\
  + mostrarEstadisticasGenerales()\\
  + escribir(String direccion, Archivo archivo)\\
  + leer(String direccion):Archivo\\
  + escrituraLectura(String direccion, Archivo archivo):Archivo\\
  + mostrarEstado():Estado\\
  + buscar(String nombre):String\\
  + comparar(Archivo arch1, Archivo arch2):Archivo\\
  + crearCarpeta(String direccion, String nombre):Bool\\
  + crearUsuario(String nombre, String userid, String pass, Nivel nivel)\\
  + gestionarPermisos(Nivel nivel, Bool[] permisos): Bool\\
  + configurarPizarra(Configuracion config)\\
  + setUserid(String userid)\\
  + setPass(String pass)\\
  - comprobarPermisos(Usuario user, int permiso):Bool\\
  - establecerConexion(Pizarra pizarra)\\
  - cerrarSesion(Pizarra pizarra) 
}

% Clase Usuario
\umlclass[x=13, y=0]{Usuario}{
  - nombre : String\\
  - userid : String\\
  - pass : String\\
  - nivel : Nivel
}{
  + comprobarUser(String pass): Bool\\
  + getNombre() : String\\
  + getUserid() : String\\
  + getNivel() : Nivel\\
  + setPass(String pass)\\
  + setNivel(Nivel nivel) 
}

% Clase Estadisticas
\umlclass[x=13, y=8]{Estadisticas}{
  - userid : String\\
  - datos[] : Integer\\
}{
  + actualizarEstadisticas(datos[] int)\\
  + getUsuario() : String\\
  + getDatos() : Integer[]\\
}

% Clase Configuracion
\umlclass[x=4, y=9.5]{Configuracion}{
  - visibilidad : Bool\\
  - IP : String\\
  - puerto : int\\
  - maxConexion : int\\
  - espacio : int\\
  - maxUser : int\\
  - depuracion : Bool\\
}{
}

% Clase Estado
\umlclass[x=-6, y=10]{Estado}{
  - ArrayList$<$Elemento$>$
}{

}

% Clase Elemento
\umlclass[x=-15, y=10]{Elemento}{
   \# nombre : String
}{

}

% Clase Carpeta
\umlclass[x=-17, y=6]{Carpeta}{
   - ArrayList$<$Elemento$>$
}{

}

% Clase Archivo
\umlclass[x=-11, y=6]{Archivo}{

}{

}

% Clase Nivel
\umlclass[x=-13, y=0]{Nivel}{
   - permisos[\$NPermisos\$] : Bool
}{
   + cambiarPermiso(int permiso, estado Bool)\\
   + resetPermisos()\\
   + setPermisos()\\
   + getPermisos():Bool[]
}

% Nota de la clase Nivel
\umlnote[x=-13, y=-5, width=5cm]{Nivel}{\textbf{\$NPERMISOS\$} es la cantidad de opciones que puede tener la plataforma, siendo cada posición una de estas opaciones}

% Relaciones

\umlinherit[geometry=-|]{Carpeta}{Elemento}

\umlcompo[mult1=0..n, mult2=1]{Elemento}{Carpeta}

\umlinherit{Archivo}{Elemento}

\umlcompo[mult2=1, mult1=0..n]{Elemento}{Estado}

\umlimpl{Pizarra}{Nivel}

\umlimpl[geometry=-|]{Pizarra}{Estado}

\umlimpl[geometry=-|]{Pizarra}{Configuracion}

\umlimpl{Pizarra}{Estadisticas}

\umlimpl{Pizarra}{Usuario}

\umlimpl{Agente}{Pizarra}

\end{tikzpicture}
\end{figure}
}


% % % % % % % % % % % % % % % % % % % % % % % % % % % % % % % % % % % % % % % % 
\usepackage[colorlinks=true,linkcolor=black,urlcolor=black,citecolor=black]{hyperref}
% % % % % % % % % % % % % % % % % % % % % % % % % % % % % % % % % % % % % % % % 

\counterwithout{figure}{part}

\setlength\parindent{0pt} % Quita la indentación

\renewcommand{\glossname}{Glosario} 

\newtheorem{teorema}{Teorema}

\makegloss

\newcommand*{\autores}{
\begin{tabular}{r l}
GII+GIS: & Jesús Alcalde Alcázar \\
		 & Germán Alonso Azcutia \\
		 & Adrián Gutiérrez Jiménez \\
GIS+MAT: & José Ignacio Escribano Pablos \\
GIS:  	 & Carlos Vázquez Sánchez
\end{tabular}
}

% Incluye archivo para utilizar el comando \portada
%*******************************************************
%                 NO MODIFICAR
\newcommand*{\FSfont}[1]{%
  \fontencoding{T1}\fontfamily{#1}\selectfont}

\newlength{\tpheight}\setlength{\tpheight}{0.9\textheight}
\newlength{\txtheight}\setlength{\txtheight}{0.9\tpheight}
\newlength{\tpwidth}\setlength{\tpwidth}{0.9\textwidth}
\newlength{\txtwidth}\setlength{\txtwidth}{0.9\tpwidth}
\newlength{\drop}
%*******************************************************

% Crea una portada con los siguientes parámetros
%
% #1 : Título 
% #2 : Subtítulo
% #3 : Subsubtítulo
% #4 : Autor(es)
% #5 : Lugar
%

\newcommand*{\portada}[5]{
\begin{titlepage}
\begingroup
\vspace*{1cm}
\drop = 0.2\txtheight
\centering
\vfill
{\Huge \scshape #1}\\[\baselineskip]
{\Large \textbf{#2}}\\[\baselineskip]
{\Large \scshape #3}\\[\baselineskip]
\vspace*{0.3cm}
{\large \textit{#4}}\\[0.5\drop]
\includegraphics[scale=0.35]{./images/logoURJC.jpg}
\vspace*{1.5cm}

{\large \scshape #5, \today} \par
\begin{center}
\end{center}
\vfill\null
\endgroup
\end{titlepage}
}
 %*****************************************************
 

\usepackage{tikz}
\usepackage{xstring}

% Aquí van todas las figuras en TikZ, hay que asignarlas un numero para poder cambiar todas a la vez.

\newcommand{\figura}[1]{
	\IfEqCase{#1}{
	% PARTE I %
		{hipervisor1}{\figurahipA}
		{hipervisor2}{\figurahipB}
		{hipervisor2real}{\figurahipBfin}
		{hardware}{\figuraHard}
		{sohost1}{\figuraSO}
		{sohost2}{\figuraSOsimpl}
		{hipervisor}{\fbox{\textcolor{red}{ESTA FIGURA HAY QUE HACERLA}}}
		{maqinavirtual}{\fbox{\textcolor{red}{ESTA FIGURA HAY QUE HACERLA}}}
		{virtualizacionnat}{\fbox{\textcolor{red}{ESTA FIGURA HAY QUE HACERLA}}}
		{virtualizacionhost}{\fbox{\textcolor{red}{ESTA FIGURA HAY QUE HACERLA}}}
		{vmware}{\fbox{\parbox{70mm}{\textcolor{red}{ESTA FIGURA HAY QUE HACERLA\\ ES COMO LA ÚLTIMA PERO LAS VM ESTAN EL EL SERVIDOR}}}}
	% PARTE II %
		{3}{\pizarra}
	}[\PackageError{figura}{Figura no declarada: #1}{}]
}


% Aqui se añaden las figuras %
\newcommand{\figurahipA}{
\begin{tikzpicture}
	\draw[rounded corners=3mm, fill=gray!30, thick] (0.05,2) rectangle (1.95,4);
	\draw[rounded corners=3mm, fill=gray!30, thick] (2.05,2) rectangle (3.95,4);
	\draw[rounded corners=3mm, fill=gray!30, thick] (6.05,2) rectangle (7.95,4);
	\draw[rounded corners=3mm, fill=gray!30, thick] (8.05,2) rectangle (9.95,4);
	\draw[rounded corners=3mm, fill={blue!40}, thick] (0,1) rectangle (10,2); 
	\draw[rounded corners=3mm, fill=yellow!40, thick] (0,0) rectangle (10,1); 

	\node at (1,3) {${VM}_1$};
	\node at (3,3) {${VM}_2$};
	\node at (5,3) {\textbf{...}};
	\node at (7,3) {${VM}_{n-1}$};
	\node at (9,3) {${VM}_n$};
	\node at (5,1.5) {\textsc{Hipervisor o VMM}};
	\node at (5,0.5) {\textsc{Hardware}};
\end{tikzpicture}
}

\newcommand{\figurahipB}{
\begin{tikzpicture}
	\draw[rounded corners=3mm, fill=gray!30, thick] (0.05,3) rectangle (1.95,5);
	\draw[rounded corners=3mm, fill=gray!30, thick] (2.05,3) rectangle (3.95,5);
	\draw[rounded corners=3mm, fill=gray!30, thick] (6.05,3) rectangle (7.95,5);
	\draw[rounded corners=3mm, fill=gray!30, thick] (8.05,3) rectangle (9.95,5);
	\draw[rounded corners=3mm, fill={blue!40}, thick] (0,2) rectangle (10,3);
	\draw[rounded corners=3mm, fill={red!40}, thick] (0,1) rectangle (10,2); 
	\draw[rounded corners=3mm, fill=yellow!40, thick] (0,0) rectangle (10,1); 

	\node at (1,4) {${VM}_1$};
	\node at (3,4) {${VM}_2$};
	\node at (5,4) {\textbf{...}};
	\node at (7,4) {${VM}_{n-1}$};
	\node at (9,4) {${VM}_n$};
	\node at (5,2.5) {\textsc{Software de virtualización}};
	\node at (5,1.5) {\textsc{Sistema operativo} {\tiny (Anfitrión)}};
	\node at (5,0.5) {\textsc{Hardware}};
\end{tikzpicture}
}

\newcommand{\figurahipBfin}{
\begin{tikzpicture}
	\draw[rounded corners=3mm, fill=green!30, thick] (0,2) rectangle (1.90,5);
	\draw[rounded corners=3mm, fill=gray!30, thick] (2.05,3) rectangle (3.95,5);
	\draw[rounded corners=3mm, fill=gray!30, thick] (6.05,3) rectangle (7.95,5);
	\draw[rounded corners=3mm, fill=gray!30, thick] (8.05,3) rectangle (9.95,5);
	\draw[rounded corners=3mm, fill={blue!40}, thick] (2,2) rectangle (10,3);
	\draw[rounded corners=3mm, fill={red!40}, thick] (0,1) rectangle (10,2); 
	\draw[rounded corners=3mm, fill=yellow!40, thick] (0,0) rectangle (10,1); 

	\node at (1,4) {Apps};
	\node at (1,3.5) {del};
	\node at (1,3) {S.O};
	\node at (3,4) {${VM}_1$};
	\node at (5,4) {\textbf{...}};
	\node at (7,4) {${VM}_{n-1}$};
	\node at (9,4) {${VM}_n$};
	\node at (6,2.5) {\textsc{Software de virtualización}};
	\node at (5,1.5) {\textsc{Sistema operativo} {\tiny (Anfitrión)}};
	\node at (5,0.5) {\textsc{Hardware}};
\end{tikzpicture}
}

\newcommand{\figuraHard}{
\begin{tikzpicture}
	\draw[rounded corners=3mm, fill=yellow!40, thick] (-0.05,-0.05) rectangle (10.05,3.05);

	\node at (2,2.5) {\textsc{Hardware Real}};
	
	\node at (1.25,1.4) {\includegraphics{./images/memoria.png}};
	\node at (1.25,0.5) {\textsc{Memoria}};
	
	\node at (3.75,1.4) {\includegraphics{./images/hdd.png}};
	\node at (3.75,0.5) {\textsc{Disco}};
	
	\node at (6.25,1.4) {\includegraphics{./images/cpu.png}};
	\node at (6.25,0.5) {\textsc{CPU}};
	
	\node at (8.75,1.4) {\includegraphics{./images/red.png}};
	\node at (8.75,0.5) {\textsc{Red}};
\end{tikzpicture}
}

\newcommand{\figuraSO}{
\begin{tikzpicture}
\definecolor{morado}{RGB}{177,100,177}
	\draw[rounded corners=3mm, fill=green!30, thick] (-.50,3) rectangle (6.5,4);
	\draw[rounded corners=3mm, fill=red!40, thick] (0,0) rectangle (6,3);
	\draw[rounded corners=3mm, fill=morado!40, thick] (0.5,0) rectangle (5.5,2);
	\draw[rounded corners=3mm, fill=blue!30, thick] (1,0) rectangle (5,1);
	\draw[rounded corners=3mm, fill=yellow!40, thick] (1.5,-1) rectangle (4.5,0);
	
	\node at (3,3.5) {\textsc{Aplicaciones}};
	\node at (2.2,2.5) {\textsc{Sistema Operativo}};
	\node at (2.5,1.5) {\textsc{Apps del SO}};
	\node at (3,0.6) {\textsc{Kernel}};
	\node at (3,-0.5) {\textsc{Hardware}};
\end{tikzpicture}
}

\newcommand{\figuraSOsimpl}{
\begin{tikzpicture}
	\draw[rounded corners=3mm, fill=green!30, thick] (-0.5,1) rectangle (6.5,2);
	\draw[rounded corners=3mm, fill=red!40, thick] (0,0) rectangle (6,1);
	\draw[rounded corners=3mm, fill=yellow!40, thick] (1.5,-1) rectangle (4.5,0);
	
	\node at (3,1.5) {\textsc{Aplicaciones}};
	\node at (3,0.5) {\textsc{Sistema Operativo}};
	\node at (3,-0.5) {\textsc{Hardware}};
\end{tikzpicture}
}


\newcommand{\pizarra}{
\begin{tikzpicture}
	\draw (0,0) rectangle (3,3);
	\draw (0.2, 0.2) rectangle (2.8,2.8);
	\draw (4.5,2.25) ellipse (1cm and 0.5cm);
	\draw (4.5,0.75) ellipse (1cm and 0.5cm);
	\draw (0.75,-1.5) ellipse (0.5cm and 1cm);
	\draw (2.25,-1.5) ellipse (0.5cm and 1cm);
	\draw (-1.5,0.75) ellipse (1cm and 0.5cm);
	\draw (-1.5,2.25) ellipse (1cm and 0.5cm);
	\draw (0.75,4.5) ellipse (0.5cm and 1cm);
	\draw (2.25,4.5) ellipse (0.5cm and 1cm);
	
	\draw[<->] (3,0.75) -- (3.5,0.75);
	\draw[<->] (3,2.25) -- (3.5,2.25);
	\draw[<->] (0.75,3) -- (0.75,3.5);
	\draw[<->] (2.25,3) -- (2.25,3.5);
	\draw[<->] (0,0.75) -- (-0.5,0.75);
	\draw[<->] (0,2.25) -- (-0.5,2.25);
	\draw[<->] (0.75,-0.5) -- (0.75,0);
	\draw[<->] (2.25,-0.5) -- (2.25,0);
	
	\node at (1.5,1.5) {\textsc{Pizarra}};
	\node at (4.5,2.25) {\textsc{Agente}};
	\node at (4.5,0.75) {\textsc{Agente}};
	\node[rotate=90] at (0.75,-1.5) {\textsc{Agente}};
	\node[rotate=90] at (2.25,-1.5) {\textsc{Agente}};
	\node at (-1.5,0.75) {\textsc{Agente}};
	\node at (-1.5,2.25) {\textsc{Agente}};
	\node[rotate=90] at (0.75,4.5) {\textsc{Agente}};
	\node[rotate=90] at (2.25,4.5) {\textsc{Agente}};
\end{tikzpicture}
}


% Reinicia el contador a 1 en cada parte
\makeatletter
\@addtoreset{chapter}{part} 
\makeatother

\begin{document}

\frontmatter % Incluye índice, portada, prefacio, ...
 
% Portada
\portada{Práctica Obligatoria}{Diseño y Arquitectura del Software}{\ }{\autores}{Móstoles}
% Fin Portada

\tableofcontents
%\thispagestyle{empty}


\chapter*{Prefacio}
\addcontentsline{toc}{chapter}{Prefacio}
\lettrine[lines=1,slope=4pt,findent=0pt]{E}{}sto es un prefacio.\\

\color{red}
FALTA:
\begin{itemize}
\item Explicar los objetivos y motivaciones de la práctica
\item Recalcar que hay dos partes y en qué se centra cada una.
\item Explicar la forma de trabajo \textit{(De forma distribuida mediante GitHub)}
\item Si se hace código, recalcarlo mucho.
\item Dar paso al inicio de la primera parte.
\end{itemize}
\color{black}

\mainmatter % Parte principal

\part{Observación}

% Añadir aqui solo los arhivos correspondientes a la parte A %

\chapter{Introducción}
\lettrine[lines=1,slope=4pt,findent=0pt]{E}{}n esta primera parte de la memoria nos vamos a encargar de hacer una descripción a nivel arquitectónico de los sistemas de virtualización de las aulas de la Universidad Rey Juan Carlos.\\

\noindent Para llevar a cabo esta tarea hemos decidido que la mejor opción pasa por describir o analizar qué es un sistema de virtualización, para después poder centrarnos en el caso concreto de las aulas de la URJC, que se encuentran virtualizados con VMware\cite{vmware} lo que significa que en el caso concreto nos centraremos en VMware, eso sí, haciendo continuas analogías a lo que se ve en las aulas.\\

\noindent Pero para entrar un poco en materia vamos a intentar explicar brevemente que es esto de la \emph{virtualización}.

\section{¿Qué es la virtualización?}
\noindent Con la intención de hacernos una idea precisa de \emph{virtualización} hemos buscado diversas definiciones, de las cuales nos quedamos con:
\begin{center}
\emph{\textquotedblleft Virtualización es la creación -a través de software- de una versión virtual de algún recurso tecnológico, como puede ser una plataforma de hardware, un sistema operativo, un dispositivo de almacenamiento u otros recursos de red.\textquotedblright}-Wikipedia\cite{defvirwiki}\\
\end{center}

\noindent Es otras palabras, la creación mediante software de elementos hardware virtuales. Un buen ejemplo de esto es cuando particionamos un disco duro; físicamente tenemos un HDD pero a nivel de software existen dos, pues el disco duro esta \emph{virtualizado} en dos particiones.\\

\noindent Dándole una vuelta de tuerca más, podemos definir \emph{virtualización} como el proceso por el cual una capa de software(VMM o \emph{Virtual Machine Monitor}) abstrae los recursos de la computadora física al Sistema Operativo o \emph{Máquina Virtual}.

\section{¿Qué es una máquina virtual?}

\noindent Al igual que antes, encontramos diversas acepciones de \emph{máquina virtual}, pero vamos a partir de la definición de \emph{Wikipedia} pues es la más completa.

\begin{center}
\emph{\textquotedblleft Una máquina virtual es un software que simula a una computadora y puede ejecutar programas como si fuese una computadora real. Este software en un principio fue definido como \textquotedblleft un duplicado eficiente y aislado de una máquina física\textquotedblright.\textquotedblright}-Wkipedia\cite{defmaqvirwiki}
\end{center}

\noindent Podemos encontrar dos tipos de éstas:

\begin{itemize}
\item \textbf{Máquinas virtuales de proceso} o máquina virtual de aplicación, se ejecuta como un proceso normal dentro de un sistema operativo y soporta un solo proceso. Su objetivo es el de proporcionar un entorno de ejecución independiente del sistema operativo y del hardware. Una máquina virtual de proceso muy popular es la de Java(\emph{Java Virtual Machine}).
\item \textbf{Máquinas virtuales de sistema} o máquinas virtuales de hardware, que permiten a la máquina física multiplicarse entre varias máquinas virtuales, cada una con su propio sistema operativo. A la capa de software que se permite la virtualización se la llama \emph{monitor de máquina virtual} o \emph{Virtual Machine Monitor}, anteriormente mencionado.
\end{itemize}

\noindent Como es obvio nosotros nos vamos a centrar en la última.

\section{Condiciones para la virtualización}

\noindent Para llevar a cabo una virtualización del sistema, Popek y Goldberg escribieron en un artículo\cite{reqvir} qué condiciones se han de dar para una virtualización eficiente, para ello dividieron el repertorio de instrucciones en:
\begin{itemize}
\item \textbf{Instrucciones privilegiadas:} Las que sólo funcionan en modo kernel y no en modo usuario.
\item \textbf{Instrucciones sensibles de control:} Las que cambian la configuración del sistema.
\item \textbf{Instrucciones sensibles de comportamiento:} Aquellas que dependen de la configuración de los recursos. 
\end{itemize}

\noindent Y como resultado de su análisis formularon estos teoremas.
\begin{teorema}
Para cualquier computadora convencional de tercera generación, se puede construir un VMM efectivo si el conjunto de instrucciones sensibles es un subconjunto de las instrucciones privilegiadas.
\end{teorema}
\begin{teorema}
Una máquina convencional de tercera generación es recursivamente virtualizable si es virtualizable y se puede construir para ella un VMM sin ninguna dependencia de sincronización.
\end{teorema}
\noindent Con esto volveremos más adelante, ahora centrémonos en su arquitectura.

\chapter[Descripción de la arquitectura]{Descripción de la arquitectura de virtualización}

\lettrine[lines=1,slope=4pt,findent=0pt]{U}{}na vez introducido el vocabulario básico y tras haber indagado un poco más en la materia vamos a centrarnos en la arquitectura como tal.\\

\section{Tipos de Virtualización}
En el apartado anterior ya se pudo vislumbrar que existen diferentes formas de virtualización, y por ello, vamos a hacer un pequeño análisis de cada uno, pero antes nos interesa conocer el término \emph{\gloss{HYP}}, ya que es el elemento central de un sistema de máquinas virtuales.

\subsection{¿Qué es un hipervisor?}
El \emph{\gloss{HYP}} o \emph{\gloss[long]{VMM}} se trata de una plataforma que permite aplicar diversas técnicas de control para utilizar, al mismo tiempo, diferentes sistemas operativos en una misma computadora.\\

Se trata de un elemento software que dependiendo de cómo se sitúe en relación con el Hardware da lugar a dos maneras diferentes de virtualizar, dos tipos de \emph{\gloss{HYP}}\cite{tipoship}:

\subsection{Hipervisor de Tipo 1 o \emph{Nativo}}
 El software del hipervisor se ubica directamente entre el hardware y las distintas máquinas virtuales, para ofrecer la funcionalidad descrita, siguiendo la siguiente estructuración:

\begin{figure}[H]
\begin{center}
\figura{1}
\end{center}
\caption[Hipervisor Tipo 1]{Esquema de un hipervisor de primer nivel}
\end{figure}

Este tipo de \emph{hipervisor} también es conocido como \emph{unhosted} o \emph{bare metal}, que en inglés hacen referencia a que no es huésped o que se ejecuta a bajo nivel, respectivamente.\\

Dentro de este tipo se encuentran VMware ESXi, VMware ESX y Microsoft Hyper-V Server, pero nos gustaría presta una atención especial a \gloss{XEN} por ser un hipervisor de código abierto desarrollado por la Universidad de Cambridge\cite{proyectoxen}\cite{proyectoxen2}.
\subsection{Hipervisor de Tipo 2 o \emph{Huésped}}
Es una arquitectura alternativa para la máquina virtual insertando una capa de virtualización encima del sistema operativo \emph{host} o huésped, siendo éste responsable de administrar el hardware. Los sistemas operativos invitados se instalarán encima del nivel de virtualización, en máquinas virtuales. Tiene la siguiente estructura:

\begin{figure}[H]
\begin{center}
\figura{2}
\end{center}
\caption[Hipervisor Tipo 2]{Esquema de un hipervisor de segundo nivel}
\end{figure}

Este tipo de hipervisor tiene una ventaja muy destacada, el usuario puede instalar esta arquitectura de máquina virtual sin modificar el sistema operativo host pudiendo descansar en el sistema operativo host para entregar los controladores de dispositivos y otros servicios de bajo nivel (se simplifica el diseño de la máquina virtual y facilita la implementación).\\

Algunos de los hipervisores tipo 2 más utilizados son: Oracle: VirtualBox, VirtualBox OSE, VMware: Workstation; siendo éste último en el que más nos vamos a centrar.

\section{Componentes}

\textcolor{red}{PEQUEÑO RESUMEN PARA INTRODUCIR QUE SE VA A HACER UN DESGLOSE DE CADA PARTE.}

\subsection{Máquina Virtual}

\textcolor{red}{AQUI VA EL ESQUEMA GENERAL DE UN VM.}


\subsection{Hipervisor}

\textcolor{red}{AQUI SE DESCRIBEN LOS COMPONENTES DE UN HIPERVISOR GENERAL.}

\section{Esquema general}

\textcolor{red}{AQUI VA EL ESQUEMA CON TODO MEZCLADO PARA QUE SE VEA A NIVEL GENERAL TODO.}


\chapter{Arquitectura y Diseño}

\lettrine[lines=1,slope=4pt,findent=0pt]{E}{}n este apartado nos centramos plenamente en lo que al diseño se refiere.\\

Los pasos a seguir son: primero describiremos la arquitectura de pizarra mediante un esquema y un diagrama UML de casos de uso; después, analizaremos el diseño en función de los requisitos obtenidos anteriormente y finalmente, diseñaremos la plataforma.

\section{Arquitectura}
Ahora vamos a centrarnos en definir gráficamente la arquitectura, en todo momento tenemos en cuenta los requisitos obtenidos en el capítulo anterior (\textit{Véase apartado \ref{reqdiseñopiz}}).

\subsection[Estructura de la arquitectura]{Estructura de nuestra arquitectura de pizarra}
Para conocer un poco más nuestra arquitectura, es necesario realizar ciertos esquemas o diagramas. Este es uno de ellos que hacen que se entienda bastante bien la arquitectura:

\begin{figure}[H]
\begin{center}
\figura{defarquitectura}
\end{center}
\caption[Estructura de la arquitectura]{Esquema donde se define la estructura básica de nuestra arquitectura de pizarra}
\end{figure}

Con esto queda claro como se organizan los agente con la pizarra, pero a grandes rasgos.

\subsection{Análisis de la estructura}
Para dar un poco más de detalle y ver de que manera se relacionan agente y pizarra, hemos construido un diagrama de casos de uso (\textit{Véase apartado \ref{casosdeuso}}) a partir de los requisitos obtenidos previamente (\textit{Véase apartado \ref{reqdiseñopiz}}), este ha sido el resultado:

\begin{figure}[H]
\centering
\casosdeuso\label{casodeuso1}
\caption{Diagrama de casos de uso de la arquitectura}
\end{figure}

Como se puede apreciar en la \emph{figura \ref{casodeuso1}} los agentes se comunican a la pizarra mediante \emph{Comprobar nivel} y \emph{Actualizar estado}, esto significa que en esas funcionalidades habrá que incluir la conexión entre ambos.\\

Por otro lado, se trata de un diagrama sencillo que no tiene gran contenido, como era de esperar.

\section{Diseño inicial o análisis}
En este apartado el principal objetivo es realizar un análisis de los requisitos obtenidos y un diseño previo, similar al realizado en el apartado anterior, sólo que en este caso de una manera más completa.\\

En primer lugar diseñaremos un diagrama de casos de uso que permite ver de una forma más detallada los requisitos extraídos anteriormente así como su relación con los distintos actores, para después analizar de forma detallada mediante diagramas de actividad cada uno de los casos de uso resultantes. Para finalizar, incluiremos un diagrama de clases que permita una visión general de las clases que componen la plataforma y su relación.\\


\subsection{Diagrama de casos de uso}\label{casosdeuso}
Un diagrama de casos de uso permite la visualización de las actividades que se permite realizar la plataforma. A estas actividades se las denomina \textit{casos de uso}. El diagrama de casos de uso define también los actores o roles que interactúan con la aplicación y las relaciones que existen entre los distintos casos de uso. Las relaciones pueden ser de dos tipos:

\begin{itemize}
	\item \textbf{Inclusión:} Un caso de uso depende del resultado de otro.
	\item \textbf{Extensión:} Un caso de uso se extiende en otros casos que, son esencialmente similares pero varían ligeramente su comportamiento.
\end{itemize}

En nuestro caso, nuestra aplicación consta de tres actores:
\begin{itemize}
\item \textbf{Usuario}
\item \textbf{Administrador:} es un usuario normal, pero con funcionalidad añadida; sólo existe uno en el sistema.
\item \textbf{Pizarra:} es el actor principal de nuestra aplicación; es el responsable de interrelacionar a los usuario y al administrador.   
\end{itemize}

Los casos de uso más importantes para los usuarios son:
\begin{itemize}
\item \textbf{Iniciar sesión} en la pizarra
\item \textbf{Ver estadísticas} 
\item \textbf{Escribir (in)} en la pizarra
\item \textbf{Leer (out)} en la pizarra
\item \textbf{Lectura/Escritura (rd)} en la pizarra
\item \textbf{Mostrar estado}
\item \textbf{Buscar} en la pizarra
\item \textbf{Comparar} archivos en la pizarra
\item \textbf{Crear carpeta} en la pizarra
\end{itemize}

Los casos de uso para el Administrador son los mismos que los del usuario y además puede:
\begin{itemize}
\item \textbf{Crear usuario} en la pizarra
\item \textbf{Gestionar permisos} de los usuarios
\item \textbf{Configurar la pizarra}
\item \textbf{Actualizar estado} de la pizarra
\end{itemize}

Por último, los casos de uso de la pizarra son:
\begin{itemize}
\item \textbf{Mostrar estado}
\item \textbf{Buscar}
\item \textbf{Comparar} archivos
\item \textbf{Crear carpeta}
\item \textbf{Gestionar permisos} de los usuarios
\end{itemize} 

\begin{sidewaysfigure}
\centering
\casos
\caption{Diagrama de casos de uso}
\end{sidewaysfigure}

\subsection{Diagramas de actividad}
Un diagrama de actividad es una representación de un proceso de forma gráfica. Consta de una serie de símbolos que representan los distintos pasos a seguir y flechas que indican el flujo de ejecución que se sigue.\\

Como es común hemos realizado un diagrama de actividad para cada uno de los casos de uso que aparecen en el diagrama anterior.

\begin{itemize}
\item \textbf{Actualizar Pizarra:} éste es uno de los estados básicos de la pizarra, ya que cada vez se escribe o se borra algún archivo hay que hacer uso de éste.\\
Los pasos que sigue son obtener los nuevos datos de la pizarra, conectarse con la base de datos, guardar estos cambios y notificarlo.

\item \textbf{Buscar:} este caso de uso busca entre los datos de la pizarra y devuelve los datos en caso de que se hayan encontrado de acuerdo a los datos de la búsqueda.\\
Los pasos que se siguen son iniciar sesión, en caso de que los datos sean correctos, se introducen los datos de la búsqueda, se muestran estos datos si se han producido resultados, se actualiza la pizarra y finalmente se notifica.

\item \textbf{Comparar:} este caso de uso compara dos o más archivos y devuelve si se ha modificado algo y qué es lo que se ha modificado.\\
Los pasos que se siguen son iniciar sesión, se introducen los datos para comparar, se muestran los resultados si los hay, se actualiza la pizarra y se notifica.

\item \textbf{Configurar pizarra:} este caso de uso sirve para configurar la pizarra, cambiando los distintos parámetros.\\
Los pasos que siguen son iniciar sesión, se configura la pizarra, se actualiza la pizarra y se notifica al usuario.

\item \textbf{Crear carpeta:} este caso de uso permite crear nuevas carpetas a los usuarios.\\
Los pasos que sigue son se inicia sesión, se crea la carpeta, se actualiza la pizarra y se notifica al usuario.

\item \textbf{Comprobar nivel:} este caso de uso permite comprobar el nivel en el que se encuentra el usuario.\\
Los pasos a seguir son obtener los datos del usuario, conectar con la base de datos, buscar la información del usuario y mostrar las opciones del nivel.

\item \textbf{Crear usuario:} este caso de uso está restringido sólo al administrador. Crea un nuevo usuario en la pizarra.\\
Los pasos a seguir son iniciar sesión, introducir los datos del usuario, conectar con la base de datos para comprobar que los datos introducidos son correctos, se crea el nuevo usuario, se actualiza la base de datos y se notifica. En caso de que no se pueda iniciar sesión o los datos introducidos no sean válidos, también se notifica.

\item \textbf{Escribir (in):} éste es uno de las operaciones básicas de la pizarra. Escribe un nuevo dato en la pizarra.\\
Los pasos a seguir son iniciar sesión, escribir los datos que se quieren que se escriban, actualizar la pizarra y notificárselo al usuario.

\item \textbf{Lectura/Escritura (rd):} este caso de uso permite tanto leer como escribir en la pizarra.\\
Los pasos a seguir son iniciar sesión, leer datos de la pizarra, obtener la posición donde se va a escribir, escribir los datos, actualizar los datos y notificar los cambios producidos.

\item \textbf{Estadísticas:} permite ver las estadísticas de un determinado usuario.\\
Los pasos a seguir son iniciar sesión, introducir datos de las estadísticas buscadas, conectar con la base de datos, buscar las estadísticas y se muestran los resultados, si los hay.

\item \textbf{Gestionar permisos:} permite cambiar los permisos tanto de lectura como de escritura de un determinado usuario.\\
Los pasos a seguir son iniciar sesión, introducir datos del usuario a cambiar los permisos, conectar con la base de datos, se busca el usuario, si existe, se gestionan los permisos y se actualiza la base de datos, en caso contrario, se notifica que el usuario no existe.

\item  \textbf{Iniciar sesión:} se introduce el id de usuario y la contraseña para poder tener acceso.\\
Los pasos a seguir son introducir los datos (id de usuario y contraseña), conectar con la base de datos, se comprueban si los datos son válidos, si lo son se establece la conexión, se comprueban los permisos y se notifica, y si no son válidos también se notifica.

\item  \textbf{Leer (out):} leer datos de la pizarra.\\
Los pasos a seguir son iniciar sesión, leer datos que se van a leer, actualizar la pizarra y se notifica.

\item \textbf{Mostrar estado:} muestra el estado actual de la pizarra.\\
Los pasos a seguir son iniciar sesión, se muestra el estado de la pizarra, se actualiza la pizarra y se notifica al usuario.
\end{itemize}

Todos los diagramas se muestran a continuación: 

\vspace*{3cm}
\begin{figure}[!h,scale=0.3]
\centering
\actualizarPizarra\label{fig:actualizarPizarra}
\caption{Actualizar pizarra}
\end{figure}
\newpage

\begin{figure}[!h]
\centering
\buscar\label{fig:buscar}
\caption{Buscar}
\end{figure}
\newpage

\begin{figure}[!h]
\centering
\comparar\label{fig:comparar}
\caption{Comparar}
\end{figure}
\newpage

\begin{figure}[!h]
\centering
\configurarPizarra\label{fig:configurarPizarra}
\caption{Configurar pizarra}
\end{figure}
\newpage

\begin{figure}[!h]
\centering
\crearCarpeta\label{fig:crearCarpeta}
\caption{Crear carpeta}
\end{figure}
\newpage

\begin{figure}[!h]
\centering
\comprobarNivel\label{fig:comprobarNivel}
\caption{Comprobar nivel}
\end{figure}
\newpage

\begin{figure}[!h]
\centering
\crearUsuario\label{fig:crearUsuario}
\caption{Crear usuario}
\end{figure}
\newpage

\begin{figure}[!h]
\centering
\escribir\label{fig:escribir}
\caption{Escribir}
\end{figure}
\newpage

\begin{figure}[!h]
\centering
\lecturaEscritura\label{fig:lecturaEscritura}
\caption{Lectura/Escritura}
\end{figure}
\newpage

\begin{figure}[!h]
\centering
\estadisticas\label{fig:estadisticas}
\caption{Estadísticas}
\end{figure}
\newpage

\begin{figure}[!h]
\centering
\gestionarPermisos\label{fig:gestionarPermisos}
\caption{Gestionar permisos}
\end{figure}
\newpage

\begin{figure}[!h]
\centering
\iniciarSesion\label{fig:iniciarSesion}
\caption{Iniciar sesión}
\end{figure}
\newpage

\begin{figure}[!h]
\centering
\leer\label{fig:leer}
\caption{Leer}
\end{figure}
\newpage

\begin{figure}[!h]
\centering
\mostrarEstado\label{fig:mostrarEstado}
\caption{Mostrar estado}
\end{figure}
\newpage

\subsection{Diagrama de clases}
Un diagrama de clases es un diagrama estático destinado a la programación orientada a objetos que permite describir las clases de un sistema, así como sus propiedades, operaciones, relaciones entre ellas y herencia.\\

A continuación detallamos cada una de las clases que aparecen en el diagrama centrándose en la funcionalidad y las relaciones entre ellas. Para una descripción más en profundidad de las operaciones que implementa ver Manual de uso.\\

\textbf{Pizarra:} El diagrama de clases se centra en esta clase. Contiene las operaciones necesarias para que los agentes puedan interactuar con ella. Almacena los datos relacionados con el estado de la pizarra, así como la lista de usuarios, las estadísticas, los permisos y la configuración.\\

\textbf{Agente:} Permite interactuar con la pizarra, dando opciones para leer o escribir en la misma, así como modificar su configuración o permisos. Hace las veces de interfaz al usuario para usar la pizarra. Contiene el nombre de usuario y la contraseña con la que se interactúa con la pizarra.\\

\textbf{Estado:} Contiene la lista de elementos.\\

\textbf{Elemento:} Puede ser de dos tipos, representado como herencia. Un archivo o una carpeta. Una carpeta contendrá a su vez un listado de elementos.\\

\textbf{Nivel: }Proporciona las operaciones necesarias para comprobar y editar los permisos de la pizarra.\\

\textbf{Configuración: }Permite la visualización y modificación de las configuraciones de la pizarra.\\

\textbf{Estadísticas: }Permite visualizar las estadísticas.\\

\textbf{Usuario:} Permite la modificación y visualización de los datos del usuario, como el nombre, el id, la contraseña y los permisos.


\begin{sidewaysfigure}
\centering
\clases
\caption{Diagrama de clases}
\end{sidewaysfigure}





\subsection{Diagramas de secuencia}

El diagrama de secuencia es un diagrama UML utilizado para modelar la interacción entre objetos. Este diagrama se modela para cada caso de uso. Consta de dos tipos de mensajes:
\begin{enumerate}
	\item \textbf{Síncrono: }Corresponden con llamadas a métodos. Se representan con flechas rellenas en negro.
	\item \textbf{Asíncrono: }Terminan inmediatamente y crean un nuevo hilo de ejecución dentro de la secuencia. Se representan con flechas sin rellenar.
\end{enumerate}

Los diagramas de secuencia correspondientes a los casos de uso de nuestra plataforma se ponen a continuación.

\begin{figure}[!h]
\centering
\seqIniciarSesion
\caption{Diagrama de secuencia de iniciar sesión}
\end{figure}

\section{Diseño final}

\chapter{Manual de uso de la librería}
\lettrine[lines=1,slope=4pt,findent=0pt]{U}{}na vez especificados los requisitos de nuestra pizarra, además de los UML necesarios que describen su funcionamiento, incorporamos un breve manual de usuario con el que un programador ajeno podría construir fácilmente una aplicación que implemente un agente de nuestra plataforma.

\section{Consideraciones previas}
Se considerará que nuestro PC cumple con todos los requisitos necesarios para que tanto nuestra librería como nuestra aplicación puedan funcionar sin problemas. Ello conlleva que se disponga de conexión a la red y tenga instalado algún compilador de C++, preferiblemente algún entorno de desarrollo como Qt\cite{QT}. En definitiva, que se pueda compilar una aplicación que incluya la plataforma.

\section{Primeros pasos} \label{sec:puesta}
En primer lugar se detallarán los primeros pasos típicos que realizará nuestra aplicación.\\

Primeramente será necesario conectarnos. Se considerará que el registro (donde se especifican tanto el nombre de usuario como la contraseña) se ha realizado anterior y correctamente, por ejemplo en nuestra aplicación).\\

\subsection{Tipos de usuarios}
Antes de comenzar es necesario recalcar que existen dos tipos de usuarios: administradores y normales. Los primeros cuentan con más opciones a la hora de organizar la pizarra, tal y como se explicará más adelante, tanto para administradores como para usuarios corrientes será iniciar sesión.

\subsection{Conectarse a la pizarra, Iniciar sesión}
Ahora se proporciona un ejemplo básico para crear un objeto agente que se conecte a la pizarra:

\begin{lstlisting}
#include <pizarra.h>
#include <iostream.h>

using namespace std;

Agente app = new Agente(loginuser, passuser, new Pizarra(IP, puerto, nombre)); 
//Creamos un agente con los parametros que hemos introducido por teclado

Usuario user = app.iniciarSesion(app.getUser(), pass); 
//Llama al metodo que nos permitira iniciar sesion

if(user != null){ 
// Si el usuario ha introducido sus datos correctamente 
// y estos se encuentran en la base de datos

}else{            
// No ha encontrado el usuario o la contrasena incorrecta, y 
// se mostrara el pertinente mensaje 

    Pizarra.existe(user) ? cout << "Invalid Password":
    cout << "Invalid user name";
}
\end{lstlisting}

\textbf{Nota: }En el ejemplo anterior se incluyen las credenciales del usuario en el código por simplificar, deben ser tratados con cuidado estos parámetros, la plataforma no se hará cargo si se pierden o son robados por culpa de una mala implementación.

A continuación se hará una breve descripción de las funcionalidades de las que dispone un agente:
\section{Funcionalidades y ejemplos}
Ahora vamos a analizar las distintas funcionalidades que se proporciona con el agente de la plataforma.\\
\subsection{Funcionalidades básica}
Aquí encontraras una lista con ejemplos de la funcionalidad más básica de la plataforma.

\subsubsection{Iniciar sesión}
Como ya se ha explicado en el apartado de primeros pasos (Véase apartado \ref{sec:puesta}) , este método será invocado constantemente por parte de los usuarios de nuestra aplicación, y su funcionamiento es el siguiente:\\

\begin{center}
\texttt{Usuario user = app.iniciarSesion(app.getUser(), pass);}
\end{center}

En primer lugar el usuario deberá introducir su nombre de usuario y contraseña formados por sendos string y si los datos con correctos, el método devolverá el usuario con sus correspondientes permisos, lo que permitirá al agente comprobar que puede y que no puede hacer.

\subsubsection{Mostrar estadísticas}\label{sec:est}
Esta funcionalidad permite saber las estadísticas a nivel de usuario, es decir, el número de archivos editados, datos borrados etc.\\
Para ello será necesario que el agente introduzca su nombre de usuario, que mediante:

\begin{center}
\texttt{Estadisticas stats = mostrarEstadisticas(String userid);}
\end{center}

Se podrán consultar los cambios realizados por cada usuario, con los modos de la clase estadísticas.

\subsubsection{Mostrar estadísticas generales}
Funciona de un modo muy parecido a mostrar estadísticas, con la diferencia de que ahora podremos consultar las estadísticas a nivel global, de todos los usuarios como un solo grupo de trabajo. Esto permitirá consultar el número de archivos subidos al repositorio, archivos borrados, etc pero todo desde un punto de vista grupal.

\begin{center}
\texttt{Estadisticas stats = mostrarEstadisticasGenerales();}
\end{center}

\subsubsection{Escribir}
Es una operación básica de la pizarra, junto a leer y lectura/escritura. Esta operación permite tanto añadir, no permite sobrescribir archivos de la pizarra. Esta operación está limitada por los permisos del usuario que quiera escribir (salvo el usuario Admin).\\

\begin{center}
\texttt{app.escribir(''. $\backslash$problemas$\backslash$tema1'', Archivo ejemplo);)}
\end{center}

Para poder escribir será necesario que el agente especifique el archivo que se quiere escrbir (\emph{''.$\backslash$problemas$\backslash$tema1$\backslash$ejemplo.getNombre()'' en nuestro caso}).\\

Debido a la importancia de esta funcionalidad se adjunta un pequeño código de ejemplo (consideramos que ya ha iniciado sesión) :
\begin{lstlisting}

. . .

if(this.user.getPermiso() == 0){ 
// El usuario dispone de los permisos necesarios

app.escribir(''.\problemas\tema1'', Archivo ejemplo) 


}else{            
// No tiene los permisos y salta un mensaje de advertencia

    Pizarra.existe(user) ? cout << "No dispone de los permisos necesarios":
}

. . .
\end{lstlisting}

\subsubsection{Lectura/Escritura}
Lectura/Escritura funciona de manera análoga a escribir, salvo que en esta operación si soporta la sobrescritura.

\begin{center}
\texttt{app.lecturaEscritura(''.$\backslash$problemas$\backslash$tema1'', Archivo ejemplo);)}
\end{center}

Por lo tanto no escribirá si el archivo ya existe.

\subsubsection{Leer}
Esta operación es análoga a escribir, salvo que en lugar de escribir se leen archivos.

\begin{center}
\texttt{Archivo ejemplo = app.lectura(''.$\backslash$problema $\backslash$tema1'');}
\end{center}

Es altamente recomendable utilizar conjuntamente los métodos \emph{leer} y \emph{lecturaEscritura} para poder editar archivos, cómo por ejemplo así:

\begin{lstlisting}

. . .

if(this.user.getPermiso() == 1 && this.user.getPermiso() == 2){ 
// El usuario dispone de los permisos necesarios

Archivo ejemplo = app.leer(''.\problemas\tema1\ej1.c'') 

\\Se edita el archivo ejemplo
      .  .  .
\\Se edita el archivo ejemplo

app.lecturaEscritura(''.\problemas\tema1'', ejemplo);

}else{            
// No tiene los permisos y salta un mensaje de advertencia

    Pizarra.existe(user) ? cout << "No dispone de los permisos necesarios":
}

. . .
\end{lstlisting}
 
\subsubsection{Mostrar estado}
Permite mostrar el estado actual de la pizarra, es decir, el número de usuarios totales, el número de archivos en la pizarra, el historial de todas las modificaciones, etc.

\begin{center}
\texttt{app.mostrarEstado();}
\end{center}

\subsubsection{Buscar}
Permite buscar archivos en la pizarra. Sólo se podrá especificar el nombre del archivo o carpeta a buscar. El agente debe introducir un String con el nombre del archivo deseado y se facilitará una lista con los archivos encontrados, en caso de que los hubiera.

\begin{center}
\texttt{app.buscar(''ejercicio1'');}
\end{center}


\subsubsection{Comparar}
Permite comparar dos archivos, pudiendo por tanto apreciar las diferencias existentes en aspectos como tamaño, formato o líneas modificadas.\\

Para ello será necesario que el agente introduzca el archivo en comparar(Archivo archivo), y al igual que en otras funcionalidades se guardará el usuario que ha realizado dicha comparación. Finalmente se devuelve un archivo donde se indican las diferencias.

\begin{center}
\texttt{Archivo diff = app.comparar(ejercicio1, solucion1);}
\end{center}

\subsubsection{Crear Carpeta}
Permite crear una carpeta dentro de la pizarra. Para ello será necesario introducir la ruta y nombre de la carpeta que deseamos crear (debe tener un nombre único).

\begin{center}
\texttt{app.crearCarpeta(''.$\backslash$ejercicios$\backslash$tema1$\backslash$soluciones'');}
\end{center}

Este método, al igual que todos requiere una comprobación de los permisos.

\subsubsection{Crear Usuario}
Permite crear nuevos agentes en la pizarra. Esta función solo puede ser usada por el Administrador.\\

\begin{center}
\texttt{app.crearUsuario(new Usuario(nombre, userid, pass, niveluser));}
\end{center}

Para ello se deberán especificar, como bien se muestra arriba, el nuevo nombre de usuario, contraseña, y nivel que dispondrá el nuevo usuario. Finalmente se comprobará que el nombre de usuario deseado no esté ya en uso, en cuyo caso se deberá cambiar.

\subsubsection{Configurar pizarra}
Mediante esta función podemos configurar algunos aspectos de nuestra pizarra, como podrían ser asuntos de conexión, máximo de conexiones simultáneas, número máximo de usuarios permitidos y algunas opciones de depuración.\\

\begin{center}
\texttt{app.configurarPizarra(new Configuracion(datos[]));}
\end{center}

\subsection{Gestionar permisos}
Una explicación más detallada precisan los permisos, que pueden tener los distintos usuarios. En nuestra aplicación los usuarios podrán tener los 7 permisos básicos típicos, tal y como se explica en la siguiente tabla: \\

\begin{center}
\begin{tabular}{|c|c|c|}
\hline
Número & Permiso  & Descripción \\
\hline
0& Leer & El usuario puede obtener información de la pizarra\\
1& Escribir & El usuario puede añadir datos a la pizarra\\
2& Lectura Escritura & El usuario puede sobrescribir datos en la pizarra\\
3& Buscar & El usuario puede buscar datos en la pizarra\\
4& Comparar & El usuario puede comparar archivos de la pizarra\\
5& Buscar & El usuario puede buscar archivos en la pizarra\\
6& Crear Carpeta & El usuario puede crear carpetas en la pizarra\\
7& Admin & El usuario que posee este permiso se comporta como admin.\\
\hline
\end{tabular}
\end{center}
\footnote{Sólo un usuario puede poseer los permisos de admin}\\

Dichos permisos sólo pueden ser modificados por el usuario Administrador mediante:

\begin{center}
\texttt{app.gestionarPermisos(usuario);}
\end{center}




\part{Diseño}
\chapter{Introducción}
\lettrine[lines=1,slope=4pt,findent=0pt]{E}{}n esta primera parte de la memoria nos vamos a encargar de hacer una descripción a nivel arquitectónico de los sistemas de virtualización de las aulas de la Universidad Rey Juan Carlos.\\

\noindent Para llevar a cabo esta tarea hemos decidido que la mejor opción pasa por describir o analizar qué es un sistema de virtualización, para después poder centrarnos en el caso concreto de las aulas de la URJC, que se encuentran virtualizados con VMware\cite{vmware} lo que significa que en el caso concreto nos centraremos en VMware, eso sí, haciendo continuas analogías a lo que se ve en las aulas.\\

\noindent Pero para entrar un poco en materia vamos a intentar explicar brevemente que es esto de la \emph{virtualización}.

\section{¿Qué es la virtualización?}
\noindent Con la intención de hacernos una idea precisa de \emph{virtualización} hemos buscado diversas definiciones, de las cuales nos quedamos con:
\begin{center}
\emph{\textquotedblleft Virtualización es la creación -a través de software- de una versión virtual de algún recurso tecnológico, como puede ser una plataforma de hardware, un sistema operativo, un dispositivo de almacenamiento u otros recursos de red.\textquotedblright}-Wikipedia\cite{defvirwiki}\\
\end{center}

\noindent Es otras palabras, la creación mediante software de elementos hardware virtuales. Un buen ejemplo de esto es cuando particionamos un disco duro; físicamente tenemos un HDD pero a nivel de software existen dos, pues el disco duro esta \emph{virtualizado} en dos particiones.\\

\noindent Dándole una vuelta de tuerca más, podemos definir \emph{virtualización} como el proceso por el cual una capa de software(VMM o \emph{Virtual Machine Monitor}) abstrae los recursos de la computadora física al Sistema Operativo o \emph{Máquina Virtual}.

\section{¿Qué es una máquina virtual?}

\noindent Al igual que antes, encontramos diversas acepciones de \emph{máquina virtual}, pero vamos a partir de la definición de \emph{Wikipedia} pues es la más completa.

\begin{center}
\emph{\textquotedblleft Una máquina virtual es un software que simula a una computadora y puede ejecutar programas como si fuese una computadora real. Este software en un principio fue definido como \textquotedblleft un duplicado eficiente y aislado de una máquina física\textquotedblright.\textquotedblright}-Wkipedia\cite{defmaqvirwiki}
\end{center}

\noindent Podemos encontrar dos tipos de éstas:

\begin{itemize}
\item \textbf{Máquinas virtuales de proceso} o máquina virtual de aplicación, se ejecuta como un proceso normal dentro de un sistema operativo y soporta un solo proceso. Su objetivo es el de proporcionar un entorno de ejecución independiente del sistema operativo y del hardware. Una máquina virtual de proceso muy popular es la de Java(\emph{Java Virtual Machine}).
\item \textbf{Máquinas virtuales de sistema} o máquinas virtuales de hardware, que permiten a la máquina física multiplicarse entre varias máquinas virtuales, cada una con su propio sistema operativo. A la capa de software que se permite la virtualización se la llama \emph{monitor de máquina virtual} o \emph{Virtual Machine Monitor}, anteriormente mencionado.
\end{itemize}

\noindent Como es obvio nosotros nos vamos a centrar en la última.

\section{Condiciones para la virtualización}

\noindent Para llevar a cabo una virtualización del sistema, Popek y Goldberg escribieron en un artículo\cite{reqvir} qué condiciones se han de dar para una virtualización eficiente, para ello dividieron el repertorio de instrucciones en:
\begin{itemize}
\item \textbf{Instrucciones privilegiadas:} Las que sólo funcionan en modo kernel y no en modo usuario.
\item \textbf{Instrucciones sensibles de control:} Las que cambian la configuración del sistema.
\item \textbf{Instrucciones sensibles de comportamiento:} Aquellas que dependen de la configuración de los recursos. 
\end{itemize}

\noindent Y como resultado de su análisis formularon estos teoremas.
\begin{teorema}
Para cualquier computadora convencional de tercera generación, se puede construir un VMM efectivo si el conjunto de instrucciones sensibles es un subconjunto de las instrucciones privilegiadas.
\end{teorema}
\begin{teorema}
Una máquina convencional de tercera generación es recursivamente virtualizable si es virtualizable y se puede construir para ella un VMM sin ninguna dependencia de sincronización.
\end{teorema}
\noindent Con esto volveremos más adelante, ahora centrémonos en su arquitectura.
\chapter[Descripción de la arquitectura]{Descripción de la arquitectura de virtualización}

\lettrine[lines=1,slope=4pt,findent=0pt]{U}{}na vez introducido el vocabulario básico y tras haber indagado un poco más en la materia vamos a centrarnos en la arquitectura como tal.\\

\section{Tipos de Virtualización}
En el apartado anterior ya se pudo vislumbrar que existen diferentes formas de virtualización, y por ello, vamos a hacer un pequeño análisis de cada uno, pero antes nos interesa conocer el término \emph{\gloss{HYP}}, ya que es el elemento central de un sistema de máquinas virtuales.

\subsection{¿Qué es un hipervisor?}
El \emph{\gloss{HYP}} o \emph{\gloss[long]{VMM}} se trata de una plataforma que permite aplicar diversas técnicas de control para utilizar, al mismo tiempo, diferentes sistemas operativos en una misma computadora.\\

Se trata de un elemento software que dependiendo de cómo se sitúe en relación con el Hardware da lugar a dos maneras diferentes de virtualizar, dos tipos de \emph{\gloss{HYP}}\cite{tipoship}:

\subsection{Hipervisor de Tipo 1 o \emph{Nativo}}
 El software del hipervisor se ubica directamente entre el hardware y las distintas máquinas virtuales, para ofrecer la funcionalidad descrita, siguiendo la siguiente estructuración:

\begin{figure}[H]
\begin{center}
\figura{1}
\end{center}
\caption[Hipervisor Tipo 1]{Esquema de un hipervisor de primer nivel}
\end{figure}

Este tipo de \emph{hipervisor} también es conocido como \emph{unhosted} o \emph{bare metal}, que en inglés hacen referencia a que no es huésped o que se ejecuta a bajo nivel, respectivamente.\\

Dentro de este tipo se encuentran VMware ESXi, VMware ESX y Microsoft Hyper-V Server, pero nos gustaría presta una atención especial a \gloss{XEN} por ser un hipervisor de código abierto desarrollado por la Universidad de Cambridge\cite{proyectoxen}\cite{proyectoxen2}.
\subsection{Hipervisor de Tipo 2 o \emph{Huésped}}
Es una arquitectura alternativa para la máquina virtual insertando una capa de virtualización encima del sistema operativo \emph{host} o huésped, siendo éste responsable de administrar el hardware. Los sistemas operativos invitados se instalarán encima del nivel de virtualización, en máquinas virtuales. Tiene la siguiente estructura:

\begin{figure}[H]
\begin{center}
\figura{2}
\end{center}
\caption[Hipervisor Tipo 2]{Esquema de un hipervisor de segundo nivel}
\end{figure}

Este tipo de hipervisor tiene una ventaja muy destacada, el usuario puede instalar esta arquitectura de máquina virtual sin modificar el sistema operativo host pudiendo descansar en el sistema operativo host para entregar los controladores de dispositivos y otros servicios de bajo nivel (se simplifica el diseño de la máquina virtual y facilita la implementación).\\

Algunos de los hipervisores tipo 2 más utilizados son: Oracle: VirtualBox, VirtualBox OSE, VMware: Workstation; siendo éste último en el que más nos vamos a centrar.

\section{Componentes}

\textcolor{red}{PEQUEÑO RESUMEN PARA INTRODUCIR QUE SE VA A HACER UN DESGLOSE DE CADA PARTE.}

\subsection{Máquina Virtual}

\textcolor{red}{AQUI VA EL ESQUEMA GENERAL DE UN VM.}


\subsection{Hipervisor}

\textcolor{red}{AQUI SE DESCRIBEN LOS COMPONENTES DE UN HIPERVISOR GENERAL.}

\section{Esquema general}

\textcolor{red}{AQUI VA EL ESQUEMA CON TODO MEZCLADO PARA QUE SE VEA A NIVEL GENERAL TODO.}

\chapter{Arquitectura y Diseño}

\lettrine[lines=1,slope=4pt,findent=0pt]{E}{}n este apartado nos centramos plenamente en lo que al diseño se refiere.\\

Los pasos a seguir son: primero describiremos la arquitectura de pizarra mediante un esquema y un diagrama UML de casos de uso; después, analizaremos el diseño en función de los requisitos obtenidos anteriormente y finalmente, diseñaremos la plataforma.

\section{Arquitectura}
Ahora vamos a centrarnos en definir gráficamente la arquitectura, en todo momento tenemos en cuenta los requisitos obtenidos en el capítulo anterior (\textit{Véase apartado \ref{reqdiseñopiz}}).

\subsection[Estructura de la arquitectura]{Estructura de nuestra arquitectura de pizarra}
Para conocer un poco más nuestra arquitectura, es necesario realizar ciertos esquemas o diagramas. Este es uno de ellos que hacen que se entienda bastante bien la arquitectura:

\begin{figure}[H]
\begin{center}
\figura{defarquitectura}
\end{center}
\caption[Estructura de la arquitectura]{Esquema donde se define la estructura básica de nuestra arquitectura de pizarra}
\end{figure}

Con esto queda claro como se organizan los agente con la pizarra, pero a grandes rasgos.

\subsection{Análisis de la estructura}
Para dar un poco más de detalle y ver de que manera se relacionan agente y pizarra, hemos construido un diagrama de casos de uso (\textit{Véase apartado \ref{casosdeuso}}) a partir de los requisitos obtenidos previamente (\textit{Véase apartado \ref{reqdiseñopiz}}), este ha sido el resultado:

\begin{figure}[H]
\centering
\casosdeuso\label{casodeuso1}
\caption{Diagrama de casos de uso de la arquitectura}
\end{figure}

Como se puede apreciar en la \emph{figura \ref{casodeuso1}} los agentes se comunican a la pizarra mediante \emph{Comprobar nivel} y \emph{Actualizar estado}, esto significa que en esas funcionalidades habrá que incluir la conexión entre ambos.\\

Por otro lado, se trata de un diagrama sencillo que no tiene gran contenido, como era de esperar.

\section{Diseño inicial o análisis}
En este apartado el principal objetivo es realizar un análisis de los requisitos obtenidos y un diseño previo, similar al realizado en el apartado anterior, sólo que en este caso de una manera más completa.\\

En primer lugar diseñaremos un diagrama de casos de uso que permite ver de una forma más detallada los requisitos extraídos anteriormente así como su relación con los distintos actores, para después analizar de forma detallada mediante diagramas de actividad cada uno de los casos de uso resultantes. Para finalizar, incluiremos un diagrama de clases que permita una visión general de las clases que componen la plataforma y su relación.\\


\subsection{Diagrama de casos de uso}\label{casosdeuso}
Un diagrama de casos de uso permite la visualización de las actividades que se permite realizar la plataforma. A estas actividades se las denomina \textit{casos de uso}. El diagrama de casos de uso define también los actores o roles que interactúan con la aplicación y las relaciones que existen entre los distintos casos de uso. Las relaciones pueden ser de dos tipos:

\begin{itemize}
	\item \textbf{Inclusión:} Un caso de uso depende del resultado de otro.
	\item \textbf{Extensión:} Un caso de uso se extiende en otros casos que, son esencialmente similares pero varían ligeramente su comportamiento.
\end{itemize}

En nuestro caso, nuestra aplicación consta de tres actores:
\begin{itemize}
\item \textbf{Usuario}
\item \textbf{Administrador:} es un usuario normal, pero con funcionalidad añadida; sólo existe uno en el sistema.
\item \textbf{Pizarra:} es el actor principal de nuestra aplicación; es el responsable de interrelacionar a los usuario y al administrador.   
\end{itemize}

Los casos de uso más importantes para los usuarios son:
\begin{itemize}
\item \textbf{Iniciar sesión} en la pizarra
\item \textbf{Ver estadísticas} 
\item \textbf{Escribir (in)} en la pizarra
\item \textbf{Leer (out)} en la pizarra
\item \textbf{Lectura/Escritura (rd)} en la pizarra
\item \textbf{Mostrar estado}
\item \textbf{Buscar} en la pizarra
\item \textbf{Comparar} archivos en la pizarra
\item \textbf{Crear carpeta} en la pizarra
\end{itemize}

Los casos de uso para el Administrador son los mismos que los del usuario y además puede:
\begin{itemize}
\item \textbf{Crear usuario} en la pizarra
\item \textbf{Gestionar permisos} de los usuarios
\item \textbf{Configurar la pizarra}
\item \textbf{Actualizar estado} de la pizarra
\end{itemize}

Por último, los casos de uso de la pizarra son:
\begin{itemize}
\item \textbf{Mostrar estado}
\item \textbf{Buscar}
\item \textbf{Comparar} archivos
\item \textbf{Crear carpeta}
\item \textbf{Gestionar permisos} de los usuarios
\end{itemize} 

\begin{sidewaysfigure}
\centering
\casos
\caption{Diagrama de casos de uso}
\end{sidewaysfigure}

\subsection{Diagramas de actividad}
Un diagrama de actividad es una representación de un proceso de forma gráfica. Consta de una serie de símbolos que representan los distintos pasos a seguir y flechas que indican el flujo de ejecución que se sigue.\\

Como es común hemos realizado un diagrama de actividad para cada uno de los casos de uso que aparecen en el diagrama anterior.

\begin{itemize}
\item \textbf{Actualizar Pizarra:} éste es uno de los estados básicos de la pizarra, ya que cada vez se escribe o se borra algún archivo hay que hacer uso de éste.\\
Los pasos que sigue son obtener los nuevos datos de la pizarra, conectarse con la base de datos, guardar estos cambios y notificarlo.

\item \textbf{Buscar:} este caso de uso busca entre los datos de la pizarra y devuelve los datos en caso de que se hayan encontrado de acuerdo a los datos de la búsqueda.\\
Los pasos que se siguen son iniciar sesión, en caso de que los datos sean correctos, se introducen los datos de la búsqueda, se muestran estos datos si se han producido resultados, se actualiza la pizarra y finalmente se notifica.

\item \textbf{Comparar:} este caso de uso compara dos o más archivos y devuelve si se ha modificado algo y qué es lo que se ha modificado.\\
Los pasos que se siguen son iniciar sesión, se introducen los datos para comparar, se muestran los resultados si los hay, se actualiza la pizarra y se notifica.

\item \textbf{Configurar pizarra:} este caso de uso sirve para configurar la pizarra, cambiando los distintos parámetros.\\
Los pasos que siguen son iniciar sesión, se configura la pizarra, se actualiza la pizarra y se notifica al usuario.

\item \textbf{Crear carpeta:} este caso de uso permite crear nuevas carpetas a los usuarios.\\
Los pasos que sigue son se inicia sesión, se crea la carpeta, se actualiza la pizarra y se notifica al usuario.

\item \textbf{Comprobar nivel:} este caso de uso permite comprobar el nivel en el que se encuentra el usuario.\\
Los pasos a seguir son obtener los datos del usuario, conectar con la base de datos, buscar la información del usuario y mostrar las opciones del nivel.

\item \textbf{Crear usuario:} este caso de uso está restringido sólo al administrador. Crea un nuevo usuario en la pizarra.\\
Los pasos a seguir son iniciar sesión, introducir los datos del usuario, conectar con la base de datos para comprobar que los datos introducidos son correctos, se crea el nuevo usuario, se actualiza la base de datos y se notifica. En caso de que no se pueda iniciar sesión o los datos introducidos no sean válidos, también se notifica.

\item \textbf{Escribir (in):} éste es uno de las operaciones básicas de la pizarra. Escribe un nuevo dato en la pizarra.\\
Los pasos a seguir son iniciar sesión, escribir los datos que se quieren que se escriban, actualizar la pizarra y notificárselo al usuario.

\item \textbf{Lectura/Escritura (rd):} este caso de uso permite tanto leer como escribir en la pizarra.\\
Los pasos a seguir son iniciar sesión, leer datos de la pizarra, obtener la posición donde se va a escribir, escribir los datos, actualizar los datos y notificar los cambios producidos.

\item \textbf{Estadísticas:} permite ver las estadísticas de un determinado usuario.\\
Los pasos a seguir son iniciar sesión, introducir datos de las estadísticas buscadas, conectar con la base de datos, buscar las estadísticas y se muestran los resultados, si los hay.

\item \textbf{Gestionar permisos:} permite cambiar los permisos tanto de lectura como de escritura de un determinado usuario.\\
Los pasos a seguir son iniciar sesión, introducir datos del usuario a cambiar los permisos, conectar con la base de datos, se busca el usuario, si existe, se gestionan los permisos y se actualiza la base de datos, en caso contrario, se notifica que el usuario no existe.

\item  \textbf{Iniciar sesión:} se introduce el id de usuario y la contraseña para poder tener acceso.\\
Los pasos a seguir son introducir los datos (id de usuario y contraseña), conectar con la base de datos, se comprueban si los datos son válidos, si lo son se establece la conexión, se comprueban los permisos y se notifica, y si no son válidos también se notifica.

\item  \textbf{Leer (out):} leer datos de la pizarra.\\
Los pasos a seguir son iniciar sesión, leer datos que se van a leer, actualizar la pizarra y se notifica.

\item \textbf{Mostrar estado:} muestra el estado actual de la pizarra.\\
Los pasos a seguir son iniciar sesión, se muestra el estado de la pizarra, se actualiza la pizarra y se notifica al usuario.
\end{itemize}

Todos los diagramas se muestran a continuación: 

\vspace*{3cm}
\begin{figure}[!h,scale=0.3]
\centering
\actualizarPizarra\label{fig:actualizarPizarra}
\caption{Actualizar pizarra}
\end{figure}
\newpage

\begin{figure}[!h]
\centering
\buscar\label{fig:buscar}
\caption{Buscar}
\end{figure}
\newpage

\begin{figure}[!h]
\centering
\comparar\label{fig:comparar}
\caption{Comparar}
\end{figure}
\newpage

\begin{figure}[!h]
\centering
\configurarPizarra\label{fig:configurarPizarra}
\caption{Configurar pizarra}
\end{figure}
\newpage

\begin{figure}[!h]
\centering
\crearCarpeta\label{fig:crearCarpeta}
\caption{Crear carpeta}
\end{figure}
\newpage

\begin{figure}[!h]
\centering
\comprobarNivel\label{fig:comprobarNivel}
\caption{Comprobar nivel}
\end{figure}
\newpage

\begin{figure}[!h]
\centering
\crearUsuario\label{fig:crearUsuario}
\caption{Crear usuario}
\end{figure}
\newpage

\begin{figure}[!h]
\centering
\escribir\label{fig:escribir}
\caption{Escribir}
\end{figure}
\newpage

\begin{figure}[!h]
\centering
\lecturaEscritura\label{fig:lecturaEscritura}
\caption{Lectura/Escritura}
\end{figure}
\newpage

\begin{figure}[!h]
\centering
\estadisticas\label{fig:estadisticas}
\caption{Estadísticas}
\end{figure}
\newpage

\begin{figure}[!h]
\centering
\gestionarPermisos\label{fig:gestionarPermisos}
\caption{Gestionar permisos}
\end{figure}
\newpage

\begin{figure}[!h]
\centering
\iniciarSesion\label{fig:iniciarSesion}
\caption{Iniciar sesión}
\end{figure}
\newpage

\begin{figure}[!h]
\centering
\leer\label{fig:leer}
\caption{Leer}
\end{figure}
\newpage

\begin{figure}[!h]
\centering
\mostrarEstado\label{fig:mostrarEstado}
\caption{Mostrar estado}
\end{figure}
\newpage

\subsection{Diagrama de clases}
Un diagrama de clases es un diagrama estático destinado a la programación orientada a objetos que permite describir las clases de un sistema, así como sus propiedades, operaciones, relaciones entre ellas y herencia.\\

A continuación detallamos cada una de las clases que aparecen en el diagrama centrándose en la funcionalidad y las relaciones entre ellas. Para una descripción más en profundidad de las operaciones que implementa ver Manual de uso.\\

\textbf{Pizarra:} El diagrama de clases se centra en esta clase. Contiene las operaciones necesarias para que los agentes puedan interactuar con ella. Almacena los datos relacionados con el estado de la pizarra, así como la lista de usuarios, las estadísticas, los permisos y la configuración.\\

\textbf{Agente:} Permite interactuar con la pizarra, dando opciones para leer o escribir en la misma, así como modificar su configuración o permisos. Hace las veces de interfaz al usuario para usar la pizarra. Contiene el nombre de usuario y la contraseña con la que se interactúa con la pizarra.\\

\textbf{Estado:} Contiene la lista de elementos.\\

\textbf{Elemento:} Puede ser de dos tipos, representado como herencia. Un archivo o una carpeta. Una carpeta contendrá a su vez un listado de elementos.\\

\textbf{Nivel: }Proporciona las operaciones necesarias para comprobar y editar los permisos de la pizarra.\\

\textbf{Configuración: }Permite la visualización y modificación de las configuraciones de la pizarra.\\

\textbf{Estadísticas: }Permite visualizar las estadísticas.\\

\textbf{Usuario:} Permite la modificación y visualización de los datos del usuario, como el nombre, el id, la contraseña y los permisos.


\begin{sidewaysfigure}
\centering
\clases
\caption{Diagrama de clases}
\end{sidewaysfigure}





\subsection{Diagramas de secuencia}

El diagrama de secuencia es un diagrama UML utilizado para modelar la interacción entre objetos. Este diagrama se modela para cada caso de uso. Consta de dos tipos de mensajes:
\begin{enumerate}
	\item \textbf{Síncrono: }Corresponden con llamadas a métodos. Se representan con flechas rellenas en negro.
	\item \textbf{Asíncrono: }Terminan inmediatamente y crean un nuevo hilo de ejecución dentro de la secuencia. Se representan con flechas sin rellenar.
\end{enumerate}

Los diagramas de secuencia correspondientes a los casos de uso de nuestra plataforma se ponen a continuación.

\begin{figure}[!h]
\centering
\seqIniciarSesion
\caption{Diagrama de secuencia de iniciar sesión}
\end{figure}

\section{Diseño final}


\backmatter % Bibliografia, lista de figuras, glosario, ...
\newpage{\ } %Añadida página para que no acabe y comience el glosario

\printgloss{./glosario}

\listoffigures

\bibliographystyle{plain}
\bibliography{./bibliografia/bibliografia}



\end{document}
