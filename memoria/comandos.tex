
% Aquí van algunos comandos para no hacer demasiado largo el preámbulo.


% Separa la lista de figuras por partes
% % % % % % % % % % % % % % % % % % % % % % % % % % % % % % % % % % % % % % % %
%
% NOTA IMPORTANTE: Este comando SIEMPRE debe ir antes de la inclusión del paquete hyperref.
%
% % % % % % % % % % % % % % % % % % % % % % % % % % % % % % % % % % % % % % % % 
% Adaptado de:
%http://tex.stackexchange.com/questions/52746/include-chapters-in-list-of-figures-with-titletoc
\makeatletter
\def\thisparttitle{}\def\thispartnumber{}
\newtoggle{noFigs}

\apptocmd{\@part}%
  {\gdef\thisparttitle{#1}\gdef\thispartnumber{\thepart}%
    \global\toggletrue{noFigs}}{}{}

% the figure environment does the job: the first time it is used after a \chapter command, 
% it writes the information of the chapter to the LoF
\AtBeginDocument{%
  \AtBeginEnvironment{figure}{%
    \iftoggle{noFigs}{
      \addtocontents{lof}{\protect\contentsline {chapter}%
        {\protect\numberline {\thispartnumber} {\thisparttitle}}{}{} }
      \global\togglefalse{noFigs}
    }{}
  }%
}

\makeatother

% Reinicia el contador a 1 en cada parte
\makeatletter
\@addtoreset{chapter}{part} 
\makeatother