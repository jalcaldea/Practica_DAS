% Lectura/escritura
\newcommand{\lecturaEscritura}{
\begin{tikzpicture}

\umlstateinitial[name=initial]

\begin{umlstate}[x=0, y=-3,name=sesion]{Iniciar sesión}
\end{umlstate}

\umlstatedecision[y=-6, name=check] 

\begin{umlstate}[x=0, y=-9,name=datos]{Leer datos de la pizarra}
\end{umlstate}

\begin{umlstate}[y=-12,name=bd]{Obtener posición escritura}
\end{umlstate}

\begin{umlstate}[y=-15,name=comprobar]{Escribir datos}
\end{umlstate}

\begin{umlstate}[x=7, y=-15, name=actualizar]{Actualizar datos}
\end{umlstate}

\begin{umlstate}[x=7, y=-12, name=notificar]{Notificar}
\end{umlstate}

\umlstatefinal[x=7, y=-9, name=final]

\umltrans{initial}{sesion}
\umltrans{sesion}{check}
\umltrans[arg1=SI]{check}{datos}
\umltrans{datos}{bd}
\umltrans{bd}{comprobar}
\umltrans{comprobar}{actualizar}
\umltrans{actualizar}{notificar}
\umltrans{notificar}{final}
\umlHVHtrans[arg1=NO, arm1=4cm,pos1=0.4]{check}{notificar}

\end{tikzpicture}
}